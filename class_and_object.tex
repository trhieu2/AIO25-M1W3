\section{What is Object-Oriented Programming?}

Object-Oriented Programming (OOP) is a programming paradigm based on the concept of "objects." These objects encapsulate data (attributes) and methods to process the data.


\subsection{Properies of OOP}
\begin{table}[!h]
	\centering
	\begin{tabular}{|p{3cm}|p{9cm}|}
		\hline
		\textbf{Property} & \textbf{Description} \\ \hline
		Abstraction & Helps to hide unnecessary details and show only the essential features of an object to the user. \\ \hline
		Inheritance & Enables code reuse by allowing a class to inherit attributes and methods from another class. \\ \hline
		Encapsulation & Protects data from unauthorized access by restricting access to certain components. \\ \hline
		Polymorphism & Allows objects to be treated differently based on the context, enabling flexible and dynamic behavior. \\ \hline
	\end{tabular}
\end{table}

\section{OOP in Python}
\subsection{Class}
A class is a blueprint for creating objects (a particular data structure), providing initial values for state (member variables or attributes), and implementations of behavior (member functions or methods).\\
\\
\\
\\
\\

\textbf{Coding}
\begin{lstlisting}[language=python, caption={Define Book class}]
	class Book:
		def __init__(self, title, author):
			self.title = title
			self.author = author
			self.rm_time = 0
			self.is_borrow = false
\end{lstlisting}

\textbf{Explanation:}
\begin{itemize}
	\item \texttt{class Book:} Defines a class named \texttt{Book}.
	\item \texttt{\_\_init\_\_ method:} Initializes the \texttt{title}, \texttt{author},\texttt{rm\_time} and \texttt{is\_borrow} attributes when a new object is created.
\end{itemize}

\subsection{Objects}
An Object is an instance of a Class. It represents a specific implementation of the class and holds its own data.

\textbf{Creating object}
\begin{lstlisting}[language=python, caption={creating object}]
	class Book:
		def __init__(self, title, author):
			self.title = title
			self.author = author
			self.rm_time = 0
			self.is_borrow = false
	
	book1 = Book("Progamming Python for beggin", "A")
	book2 = Book("Data structure","B")
	
	print(book1.title)
	print(book2.author)
\end{lstlisting}

\begin{lstlisting}[language=python,caption={Output object}]
	Progamming Python for beggin
	B
\end{lstlisting}

\textbf{Explanation:}
\begin{itemize}
	\item \texttt{book1 = Book("Programming Python for Beginners", "A"):} Creates an object of the \texttt{Book} class with \texttt{title} as "Programming Python for Beginners" and \texttt{author} as "A".
	\item \texttt{book2 = Book("Data Structure", "B"):} Creates another object of the \texttt{Book} class with \texttt{title} as "Data Structure" and \texttt{author} as "B".
	\item \texttt{book1.title:} Accesses the \texttt{title} attribute of the \texttt{book1} object.
	\item \texttt{book2.author:} Accesses the \texttt{author} attribute of the \texttt{book2} object.
\end{itemize}

\subsection{Sefl parameter}
\begin{itemize}
	\item The self keyword is used to represent the instance of the class.
	\item Variables prefixed with self are the attributes of the class, while others are merely local variables of the class
\end{itemize}

\begin{lstlisting}
	class Book:
		def __init__(self, title, author):
			# self-prefixed attributes (instance variables)
			self.title = title
			self.author = author
			self.status = "Available"  # Default value
			# Local variable
			temp = "This is temporary"  # Only accessible within this method
			
		def borrow(self):
			if self.status == "Available":
			self.status = "Borrowed"
			print(f"The book '{self.title}' has been borrowed.")
			else:
			print(f"The book '{self.title}' is already borrowed.")
		
			# temp is not accessible here
			# print(temp)  # This will raise a NameError
	
	book1 = Book("Data Structures", "A")
	# print(book1.temp) # This will raise a AttributeError
\end{lstlisting}


\subsection{\_\_init\_\_ method}
\_\_init\_\_ method is the constructor in Python, automatically called when a new object is created. It initializes the attributes of the class.
\begin{lstlisting}[language=python, caption={\_\_init\_\_ method}]
	class Birthday:
		def __init__(self, day, month,year):
			self.day = day
			self.month = month
			self.year = year
			
	birthday_peter = Birthday(1,1,2010)
	print(f"{birthday_peter.day}/{birthday_peter.month}/{birthday_peter.year}")
\end{lstlisting}

\begin{lstlisting}[language=python,caption={Output \_\_init\_\_ method}]
	1/1/2010
\end{lstlisting}
\textbf{Explanation:}
\begin{itemize}
	\item \texttt{\_\_init\_\_} : Special method used for initialization.
	\item \texttt{self.day}, \texttt{self.month}, \texttt{self.year:}  Instance attributes initialized in the constructor.
\end{itemize}

\subsection{\_\_call\_\_ Method}
The \_\_call\_\_ method in Python allows an object of a class to be called like a function. It is automatically executed when the object is followed by parentheses.

\begin{lstlisting}[language=python, caption={\_\_call\_\_ Method}]
	class Greeting:
	def __init__(self, name):
	self.name = name
	
	def __call__(self, message):
	return f"{message}, {self.name}!"
	
	greet = Greeting("Alice")
	print(greet("Hello"))
\end{lstlisting}

\begin{lstlisting}[language=python, caption={Output \_\_call\_\_ method}]
	Hello, Alice!
\end{lstlisting}

\textbf{Explanation:}
\begin{itemize}
	\item \texttt{\_\_call\_\_}: A special method in Python that makes an instance of a class callable like a function.
	\item \texttt{self.name}: An instance attribute initialized in the constructor, storing the name.
	\item \texttt{greet}: An instance of the \texttt{Greeting} class, initialized with the name "Alice".
	\item \texttt{greet("Hello")}: Invokes the \_\_call\_\_ method of the \texttt{greet} object, returning "Hello, Alice!".
\end{itemize}

\section{Python Inheritance}
Inheritance is a fundamental concept in object-oriented programming (OOP). It allows one class (child class) to inherit the attributes and methods of another class (parent class). This promotes code reuse and enables the creation of a hierarchical relationship between classes.

\subsection{How Inheritance Works}
When a class inherits from another:
\begin{itemize}
	\item The child class gains access to all public and protected attributes and methods of the parent class.
	\item The child class can also override methods of the parent class to provide specific behavior.
	\item The \texttt{super()} function is used to access methods and attributes of the parent class from the child class.
\end{itemize}

\subsection{Example: Basic Inheritance}
\begin{lstlisting}[language=python, caption={Inheritance Example}]
	# Parent class
	class Animal:
		def __init__(self, name):
			self.name = name
	
		def speak(self):
			return f"{self.name} makes a sound."
	
	# Child class
	class Dog(Animal):
		def __init__(self, name, breed):
			super().__init__(name)  # Initialize the parent class
			self.breed = breed
	
		def speak(self):
			return f"{self.name}, the {self.breed}, barks."
	
	# Child class
	class Cat(Animal):
		def __init__(self, name, color):
			super().__init__(name)  # Initialize the parent class
			self.color = color
		
		def speak(self):
			return f"{self.name}, the {self.color} cat, meows."
	
	# Create instances
	dog = Dog("Buddy", "Golden Retriever")
	cat = Cat("Whiskers", "white")
	
	print(dog.speak())  # Output: Buddy, the Golden Retriever, barks.
	print(cat.speak())  # Output: Whiskers, the white cat, meows.
\end{lstlisting}

\subsection{Explanation}
\begin{itemize}
	\item \texttt{class Animal:} This is the parent class. It contains common attributes (\texttt{name}) and methods (\texttt{speak}) that all child classes can inherit.
	\item \texttt{super().\_\_init\_\_(name):} The \texttt{super()} function is used in the child class to call the parent class's constructor and initialize inherited attributes.
	\item \texttt{class Dog(Animal):} This is a child class inheriting from \texttt{Animal}. It adds a new attribute (\texttt{breed}) and overrides the \texttt{speak()} method.
	\item \texttt{class Cat(Animal):} Similar to \texttt{Dog}, this child class inherits from \texttt{Animal}, adds a new attribute (\texttt{color}), and overrides the \texttt{speak()} method.
	\item \texttt{dog.speak():} Calls the overridden \texttt{speak()} method in the \texttt{Dog} class, which includes specific behavior for dogs.
	\item \texttt{cat.speak():} Calls the overridden \texttt{speak()} method in the \texttt{Cat} class, which includes specific behavior for cats.
\end{itemize}






